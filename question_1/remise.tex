\documentclass[12pt,letterpaper]{article}
\usepackage[utf8]{inputenc}
\usepackage[french]{babel}
\usepackage{fullpage}
\usepackage[noline, french, onelanguage, noend]{algorithm2e}
\usepackage[top=2cm, bottom=4.5cm, left=2.5cm, right=2.5cm]{geometry}
\usepackage{amsmath,amsthm,amsfonts,amssymb,amscd}
\usepackage{lastpage}
\usepackage{enumerate}
\usepackage{fancyhdr}
\usepackage{mathrsfs}
\usepackage{xcolor}
\usepackage{multicol}
\usepackage{graphicx}
\usepackage{listings}
\usepackage{hyperref}
\usepackage{mathtools}
\DeclarePairedDelimiter{\ceil}{\lceil}{\rceil}
\DeclarePairedDelimiter\floor{\lfloor}{\rfloor}

\hypersetup{%
  colorlinks=true,
  linkcolor=blue,
  linkbordercolor={0 0 1}
}
\newcommand{\N}{\mathbb{N}}
\newcommand{\algskip}{\vspace{3pt}}
\renewcommand{\O}{\mathcal{O}}
\renewcommand\lstlistingname{Algorithm}
\renewcommand\lstlistlistingname{Algorithms}
\def\lstlistingautorefname{Alg.}

\lstdefinestyle{Python}{
    language        = Python,
    frame           = lines, 
    basicstyle      = \footnotesize,
    keywordstyle    = \color{blue},
    stringstyle     = \color{green},
    commentstyle    = \color{red}\ttfamily
}

\setlength{\parindent}{0.0in}
\setlength{\parskip}{0.05in}

% Edit these as appropriate
\newcommand\course{IFT436}
\newcommand\hwnumber{6}                  % <-- homework number
\newcommand\NetIDa{boue2327}           % <-- NetID of person #1
\newcommand\NetIDb{valr2802}           % <-- NetID of person #2 (Comment this line out for problem sets)

\pagestyle{fancyplain}
\headheight 35pt
\lhead{\NetIDa}
\lhead{\NetIDa\\\NetIDb}                 % <-- Comment this line out for problem sets (make sure you are person #1)
\chead{\textbf{\Large Devoir \hwnumber}}
\rhead{\course \\ 5 décembre 2019}
\lfoot{}
\cfoot{}
\rfoot{\small\thepage}
\headsep 1.5em

\begin{document}

\section*{Question 1}

(a) L'algorithme est de Monte Carlo, puisque le temps d'exécution est borné par la valeur de $k$, permettant alors de possiblement retourner une valeur erronée.\newline

(b) Il y a trois situations à prendre en compte : \newline

1 - La probabilité que toutes les valeurs soient différentes est de $\frac{3!}{27}$\newline
2 - La probabilité que toutes les valeurs soient 42 est de : $\frac{1}{27}$\newline
3 - La probabilité que 42 se retrouve dans la liste est de : $\frac{3(1\cdot 2!)}{27}$\newline

Le $27$ vient de $3^3$. Donc la probabilité totale est de $\frac{13}{27}$.\newline

(c) Si $k = 1$, alors il y a $\frac{1}{2}$ probabilité de faire un erreur, car on ne choisi qu'un seul élément de $s$, et la moitié des éléments de $s$ ne sont pas raisonnables.\newline

Si $k = 3$, il y a 4 possiblités d'erreur ("-" et "+" signifie rang non acceptable (trop petit ou trop grand) pour être raisonnable):\newline

1 : -, -, - où la probabilité est de $(\frac{1}{4})^3$\newline
2 : -, -, x (où x est soit rang acceptable ou "+") où la probabilité est de $\dbinom{3}{2} \cdot (\frac{1}{4})^2 \cdot \frac{3}{4}$\newline
3 : +, +, + où la probabilité est de $(\frac{1}{4})^3$\newline
4 : x, +, + (où x est soit rang acceptable ou "-") où la probabilité est de $\dbinom{3}{2} \cdot (\frac{1}{4})^2 \cdot \frac{3}{4}$\newline

Le total est donc $\frac{9}{64} + \frac{1}{64} + \frac{9}{64} + \frac{1}{64} = 2 \cdot (\frac{9}{64} + \frac{1}{64}) = \frac{5}{16}$.\newline

Si $k = k$, alors la formule est : $2 \cdot \sum_{n = \ceil*{\frac{k}{2}}}^{k} \dbinom{k}{n} \cdot (\frac{1}{4})^n \cdot (\frac{3}{4})^{k-n}$. La formule suit la même logique que lorsque $k = 3$.\newline

En effet, le $n$ parmi $k$ représente le nombre de combinaisons de soit + ou -, le $(\frac{1}{4})^n$ représente la possibilité d'obtenir les éléments non responsable et le $(\frac{3}{4})^{k-n}$ représente la possibilité de l'autre ou des autres élément(s). La sommation indique que le + ou - se promène, vers la droite de la séquence, puisque qu'il y a erreur seulement s'il y a un élément - ou + à $t[\ceil*{\frac{k}{2}}]$, ce qui est la raison de $n = \ceil*{\frac{k}{2}}$. Finalement, la multiplication par $2$ représente le fait que c'est le calcul sera le même pour le cas de + et le cas de -.
\newpage
(d)\newline
\begin{algorithm}[H]
\DontPrintSemicolon
\LinesNumbered
\KwIn{Élément $x \in s$ et la séquence $s$}
\KwResult{Si $x$ est un élément raisonnable de $s$}
\algskip

\SetKwFunction{algo}{algo}
\SetKwProg{myproc}{}{}{}
\myproc{\algo{$x, s$}$\mathtt{:}$}{
	position $\leftarrow 1$\;	
	\For{élément $i \in s$}{
		\If{$i < x$}{
			$position \leftarrow position + 1$
		}
	}
	\If{position $> \ceil*{\frac{|s|}{4}} \land $ position $\leq \floor*{\frac{3 \cdot |s|}{4}}$}{
		\Return{vrai}	
	}\Else{
		\Return{faux}
	}
}
\end{algorithm}

(e) On a calculé en (c), que la probabilité d'erreur pour $k=3$ était de $\frac{5}{16}$, donc la probabilité de succès est de $1 - \frac{5}{16} = \frac{11}{16}$. L'espérance est donc de $\frac{1}{p} = \frac{1}{\frac{11}{16}} = \frac{16}{11}$, donc l'algorithme fonctionnera en $\O(\frac{16}{11}n) = \O(n)$.
\end{document}
